\chapter{System Maintenance}

\section{Environment}

\subsection{Software}
The software and modules I used to implement the program:
\begin{itemize}
	\item Python 3
	\item IDLE
	\item PyQt 4
	\item SQLite3
	\item Notepad++
	\item 
	\item 
	\item 
\end{itemize}

\subsection{Usage Explanation}
\begin{center}
	\begin{tabular}{|p{2cm}|p{2cm}|p{2cm}|p{2cm}|l}
		\hline
		\textbf{Software}   & \textbf{Explaination for use of software} \\ \hline
		Python 3 & I know this language well and python is a relatively powerful language. Python allows bug finding and fixing to be incredibly easy as the program can be compiled quickly as you write it instead of using a seperate compiler. \\ \hline
		IDLE &  I used this python environment as it comes with the basic installtion of python and is the only editor installed on school computers.
		PyQt4 & PyQt allows for the easy creation of a GUI
		SQLite3 & Used for database creation and editing as I have been taught this and SQL allows for easy database manipulation
		Notepad++ & Used for the creation of HTML files as I know how to use it and HTML files can be run from the program
\end{tabular}
\end{center}


\subsection{Features Used}
\begin{center}
	\begin{tabular}{|p{2cm}|p{2cm}|p{2cm}|p{2cm}|l}
		\hline
		\textbf{Software}   & \textbf{Features used} \\ \hline
		Python 3 & I know this language well and python is a relatively powerful language.  \\ \hline
		IDLE &  I used this python environment as it comes with the basic installtion of python and is the only editor installed on school computers.
		PyQt4 & PyQt allows for the easy creation of a GUI
		SQLite3 & Used for database creation and editing as I have been taught this and SQL allows for easy database manipulation
		Notepad++ & Used for the creation of HTML files as I know how to use it and HTML files can be run from the program
\end{tabular}
\end{center}


\section{System Overview}

%use as many subsections as necessary for the system components
\subsection{System Component}

\section{Code Structure}

%use as many subsections as necessary for the code sections
\subsection{Particular Code Section}
%the code below can be uncommented and used to get a code section from a particular file
\begin{comment}
\begin{figure}[H]
    \pythonfile[firstline=5,lastline=10]{./tex/function_programs/print_function.py}
    \caption{The print() function} \label{fig:print_function}
\end{figure}
\end{comment}

\section{Variable Listing}

\section{System Evidence}

\subsection{User Interface}

\subsection{ER Diagram}

\subsection{Database Table Views}

\subsection{Database SQL}

\subsection{SQL Queries}

\section{Testing}

\subsection{Summary of Results}

\subsection{Known Issues}

\section{Code Explanations}

\subsection{Difficult Sections}

\subsection{Self-created Algorithms}

\section{Settings}

\section{Acknowledgements}

\section{Code Listing}
\begin{landscape}
%include as many subsections as you have modules
\subsection{Module 1}
%the code below can be uncommented and used to get a code section from a particular file
\begin{comment}
\pythonfile[firstline=5]{./tex/function_programs/print_function.py}
\end{comment}
\end{landscape}
